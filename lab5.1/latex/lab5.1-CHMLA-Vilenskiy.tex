\documentclass[a4paper, 14pt]{extarticle}

% Поля
%--------------------------------------
\usepackage{geometry}
\geometry{a4paper,tmargin=2cm,bmargin=2cm,lmargin=3cm,rmargin=1cm}
%--------------------------------------


%Russian-specific packages
%--------------------------------------
\usepackage[T2A]{fontenc}
\usepackage[utf8]{inputenc} 
\usepackage[english, main=russian]{babel}
%--------------------------------------

\usepackage{textcomp}

% Красная строка
%--------------------------------------
\usepackage{indentfirst}               
%--------------------------------------             


%Graphics
%--------------------------------------
\usepackage{graphicx}
\graphicspath{ {./images/} }
\usepackage{wrapfig}
%--------------------------------------

% Полуторный интервал
%--------------------------------------
\linespread{1.3}                    
%--------------------------------------

%Выравнивание и переносы
%--------------------------------------
% Избавляемся от переполнений
\sloppy
% Запрещаем разрыв страницы после первой строки абзаца
\clubpenalty=10000
% Запрещаем разрыв страницы после последней строки абзаца
\widowpenalty=10000
%--------------------------------------

%Списки
\usepackage{enumitem}

%Подписи
\usepackage{caption} 

%Гиперссылки
\usepackage{hyperref}

\hypersetup {
	unicode=true
}

%Рисунки
%--------------------------------------
\DeclareCaptionLabelSeparator*{emdash}{~--- }
\captionsetup[figure]{labelsep=emdash,font=onehalfspacing,position=bottom}
%--------------------------------------

\usepackage{tempora}

%Листинги
%--------------------------------------
\usepackage{listings}
\lstset{
  basicstyle=\ttfamily\footnotesize, 
  %basicstyle=\footnotesize\AnkaCoder,        % the size of the fonts that are used for the code
  breakatwhitespace=false,         % sets if automatic breaks shoulbd only happen at whitespace
  breaklines=true,                 % sets automatic line breaking
  captionpos=t,                    % sets the caption-position to bottom
  inputencoding=utf8,
  frame=single,                    % adds a frame around the code
  keepspaces=true,                 % keeps spaces in text, useful for keeping indentation of code (possibly needs columns=flexible)
  keywordstyle=\bf,       % keyword style
  numbers=left,                    % where to put the line-numbers; possible values are (none, left, right)
  numbersep=5pt,                   % how far the line-numbers are from the code
  xleftmargin=25pt,
  xrightmargin=25pt,
  showspaces=false,                % show spaces everywhere adding particular underscores; it overrides 'showstringspaces'
  showstringspaces=false,          % underline spaces within strings only
  showtabs=false,                  % show tabs within strings adding particular underscores
  stepnumber=1,                    % the step between two line-numbers. If it's 1, each line will be numbered
  tabsize=2,                       % sets default tabsize to 8 spaces
  title=\lstname                   % show the filename of files included with \lstinputlisting; also try caption instead of title
}
%--------------------------------------

%%% Математические пакеты %%%
%--------------------------------------
\usepackage{amsthm,amsfonts,amsmath,amssymb,amscd}  % Математические дополнения от AMS
\usepackage{mathtools}                              % Добавляет окружение multlined
\usepackage[perpage]{footmisc}
%--------------------------------------

%--------------------------------------
%			НАЧАЛО ДОКУМЕНТА
%--------------------------------------

\begin{document}

%--------------------------------------
%			ТИТУЛЬНЫЙ ЛИСТ
%--------------------------------------
\begin{titlepage}
\thispagestyle{empty}
\newpage


%Шапка титульного листа
%--------------------------------------
\vspace*{-60pt}
\hspace{-65pt}
\begin{minipage}{0.3\textwidth}
\hspace*{-20pt}\centering
\includegraphics[width=\textwidth]{emblem}
\end{minipage}
\begin{minipage}{0.67\textwidth}\small \textbf{
\vspace*{-0.7ex}
\hspace*{-6pt}\centerline{Министерство науки и высшего образования Российской Федерации}
\vspace*{-0.7ex}
\centerline{Федеральное государственное бюджетное образовательное учреждение }
\vspace*{-0.7ex}
\centerline{высшего образования}
\vspace*{-0.7ex}
\centerline{<<Московский государственный технический университет}
\vspace*{-0.7ex}
\centerline{имени Н.Э. Баумана}
\vspace*{-0.7ex}
\centerline{(национальный исследовательский университет)>>}
\vspace*{-0.7ex}
\centerline{(МГТУ им. Н.Э. Баумана)}}
\end{minipage}
%--------------------------------------

%Полосы
%--------------------------------------
\vspace{-25pt}
\hspace{-35pt}\rule{\textwidth}{2.3pt}

\vspace*{-20.3pt}
\hspace{-35pt}\rule{\textwidth}{0.4pt}
%--------------------------------------

\vspace{1.5ex}
\hspace{-35pt} \noindent \small ФАКУЛЬТЕТ\hspace{80pt} <<Информатика и системы управления>>

\vspace*{-16pt}
\hspace{47pt}\rule{0.83\textwidth}{0.4pt}

\vspace{0.5ex}
\hspace{-35pt} \noindent \small КАФЕДРА\hspace{50pt} <<Теоретическая информатика и компьютерные технологии>>

\vspace*{-16pt}
\hspace{30pt}\rule{0.866\textwidth}{0.4pt}
  
\vspace{11em}

\begin{center}
\Large {\bf Лабораторная работа № 5.1} \\ 
\large {\bf по курсу <<Численные методы линейной алгебры>>} \\
\large <<Вычисление собственных значений и собственных векторов симметричной матрицы методом А.М. Данилевского>> 
\end{center}\normalsize

\vspace{8em}


\begin{flushright}
  {Студент группы ИУ9-72Б Виленский С. Д. \hspace*{15pt}\\ 
  \vspace{2ex}
  Преподаватель Посевин Д. П.\hspace*{15pt}}
\end{flushright}

\bigskip

\vfill
 

\begin{center}
\textsl{Москва 2024}
\end{center}
\end{titlepage}
%--------------------------------------
%		КОНЕЦ ТИТУЛЬНОГО ЛИСТА
%--------------------------------------

\renewcommand{\ttdefault}{pcr}

\setlength{\tabcolsep}{3pt}
\newpage
\setcounter{page}{2}

\section{Задание}\label{Sect::task}

Реализовать метод поиска собственных значений действительной симметричной матрицы A размером 4х4. 
Проверить корректность вычисления собственных значений по теореме Виета.
Проверить выполнение условий теоремы Гершгорина о принадлежности собственных значений соответствующим объединениям кругов Гершгорина.
Вычислить собственные вектора и проверить выполнение условия ортогональности собственных векторов.
Проверить решение на матрице приведенной в презентации.
Продемонстрировать работу приложения для произвольных симметричных матриц размером n х n с учетом выполнения пунктов приведенных выше.

\section{Результаты}\label{Sect::res}

Исходный код программы представлен в листингах~\ref{lst:code1}-~\ref{lst:code3}.

\begin{figure}[!htb]
\begin{lstlisting}[caption={Код},label={lst:code1}]
using LinearAlgebra
using PolynomialRoots

function I_matrix(n::Int)::Matrix{Float64}
    I = Matrix{Float64}(zeros(n, n))
    for i in 1:n
        I[i, i] = 1.0
    end
    return I
end

function danilevsky_method(A::Matrix{Float64})
    n = size(A, 1)

    B_i = Vector{Matrix{Float64}}(undef, n - 1)
    D = copy(A)

    for k in n:-1:2
        B_inv = I_matrix(n)
        B_inv[k - 1, :] = D[k, :]
        B_i[n - k + 1] = inv(B_inv)

        D = B_inv * D * B_i[n - k + 1]
    end
    P = D[1, :]
    eigen_vals = real.(roots(push!(-reverse(P), 1)))

    B = I_matrix(n)
    for B_ in B_i
\end{lstlisting}
\end{figure}
\begin{figure}[!htb]
\begin{lstlisting}[caption={Код},label={lst:code2}]
        B *= B_
    end

    y_i = [[eigen_val ^ i for i in (n-1):-1:0] for eigen_val in eigen_vals]
    
    x_i = [B * y for y in y_i]

    return eigen_vals, normalize.(x_i)
end

function check_by_Viet(A::Matrix{Float64}, eigen_vals::Vector{Float64})::Float64
    return abs(tr(A) - sum(eigen_vals))
end

function check_by_Gershgorin(A::Matrix{Float64}, eigen_vals::Vector{Float64})::Bool
    n = size(A, 1)

    start_union = undef
    stop_union = undef
    for i in 1:n
        diag_elem = A[i, i]
        line_sum = sum(abs.(A[i, :])) - abs(diag_elem)

        start = diag_elem - line_sum
        if start_union == undef || start < start_union
            start_union = start
        end
        
        stop = diag_elem + line_sum
        if stop_union == undef || stop > start_union
            stop_union = stop
        end
    end
    
    return all(start_union <= eigen_val <= stop_union for eigen_val in eigen_vals)
end

function check_ortogonal(eigen_vectors::Vector{Vector{Float64}})::Bool
    n = size(eigen_vectors, 1)
    for i in 1:(n-1)
        for j in (i+1):n
            if abs(eigen_vectors[i]'eigen_vectors[j]) > 1e-5
                return false
            end
        end
    end
    return true
end

A = [
    2.2 1.0 0.5 2.0;
\end{lstlisting}
\end{figure}
\begin{figure}[!htb]
\begin{lstlisting}[caption={Код},label={lst:code3}]
    1.0 1.3 2.0 1.0;
    0.5 2.0 0.5 1.6;
    2.0 1.0 1.6 2.0
]

eigen_vals, eigen_vectors = danilevsky_method(A)
println("Eigenvalues: ", eigen_vals)
error_viet = check_by_Viet(A, eigen_vals)
println("Error in calculating eigenvalues using Vieta's theorem: ", error_viet)
check = check_by_Gershgorin(A, eigen_vals)
println("Verification of Gershgorin's theorem: ", check)
println("Eigenvectors:")
for eigen_vector in eigen_vectors
    println(eigen_vector)
end
check_vects = check_ortogonal(eigen_vectors)
println("Orthogonality of eigenvectors: ", check_vects)

n = 5
A = Matrix{Float64}(Symmetric(rand(-10.0:0.1:10.0,n,n)))
eigen_vals, eigen_vectors = danilevsky_method(A)
println("Eigenvalues: ", eigen_vals)
error_viet = check_by_Viet(A, eigen_vals)
println("Error in calculating eigenvalues using Vieta's theorem: ", error_viet)
check = check_by_Gershgorin(A, eigen_vals)
println("Verification of Gershgorin's theorem: ", check)
println("Eigenvectors:")
for eigen_vector in eigen_vectors
    println(eigen_vector)
end
check_vects = check_ortogonal(eigen_vectors)
println("Orthogonality of eigenvectors: ", check_vects)
\end{lstlisting}
\end{figure}

Результат работы программы представлен в листингах~\ref{lst:code4}-~\ref{lst:code5}.

\begin{figure}[!htb]
\begin{lstlisting}[caption={Результат работы программы},label={lst:code4}]
Eigenvalues: [5.652032331764589, -1.420086593950619, 1.5454183350534156, 0.22263592713261507]
Error in calculating eigenvalues using Vieta's theorem: 1.7763568394002505e-15
Verification of Gershgorin's theorem: true
Eigenvectors:
[0.5317360693095499, 0.44619412190869223, 0.40881553418500616, 0.5924841071103837]
[-0.2220428365454722, 0.5159103236551117, -0.7572742312071333, 0.3332705439047439]
[0.62892976467108, -0.5725742255591189, -0.48565379676310105, 0.2018576157239048]
[-0.5219205710113896, -0.45486932161400195, 0.1534470183752563, 0.705086399217363]
Orthogonality of eigenvectors: true
\end{lstlisting}
\end{figure}
\begin{figure}[!htb]
\begin{lstlisting}[caption={Результат работы программы},label={lst:code5}]
Eigenvalues: [18.176061276778213, -8.924899361398401, 5.39378895734575, 1.874589613387166, -0.5195404861127171]
Error in calculating eigenvalues using Vieta's theorem: 1.2434497875801753e-14
Verification of Gershgorin's theorem: false
Eigenvectors:
[0.3996021709709015, -0.665653553721814, 0.5558056083971021, -0.17477433349077834, 0.2403279205586481]
[-0.5492894758466208, -0.10766023132451748, 0.24684288967053, 0.6180553989016193, 0.4937271088597009]
[-0.5156957615883632, -0.37662692283972776, -0.392079500533507, -0.5966956605656002, 0.28712017761845604]
[-0.2019522237982415, 0.5759766479137343, 0.5948706624020939, -0.48096669550969695, 0.20558728848053778]
[0.48153126322413037, 0.2678015740412564, -0.350073240577105, 0.009315724023380937, 0.7574773283713271]
Orthogonality of eigenvectors: true
\end{lstlisting}
\end{figure}

\section{Выводы}\label{Sect::conclusion}

Метод Данилевского нахождения совбственных значений и векторов симметричной матрицы является достаточно точным.

\end{document}
