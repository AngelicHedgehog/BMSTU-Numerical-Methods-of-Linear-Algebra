\documentclass[a4paper, 14pt]{extarticle}

% Поля
%--------------------------------------
\usepackage{geometry}
\geometry{a4paper,tmargin=2cm,bmargin=2cm,lmargin=3cm,rmargin=1cm}
%--------------------------------------


%Russian-specific packages
%--------------------------------------
\usepackage[T2A]{fontenc}
\usepackage[utf8]{inputenc} 
\usepackage[english, main=russian]{babel}
%--------------------------------------

\usepackage{textcomp}

% Красная строка
%--------------------------------------
\usepackage{indentfirst}               
%--------------------------------------             


%Graphics
%--------------------------------------
\usepackage{graphicx}
\graphicspath{ {./images/} }
\usepackage{wrapfig}
%--------------------------------------

% Полуторный интервал
%--------------------------------------
\linespread{1.3}                    
%--------------------------------------

%Выравнивание и переносы
%--------------------------------------
% Избавляемся от переполнений
\sloppy
% Запрещаем разрыв страницы после первой строки абзаца
\clubpenalty=10000
% Запрещаем разрыв страницы после последней строки абзаца
\widowpenalty=10000
%--------------------------------------

%Списки
\usepackage{enumitem}

%Подписи
\usepackage{caption} 

%Гиперссылки
\usepackage{hyperref}

\hypersetup {
	unicode=true
}

%Рисунки
%--------------------------------------
\DeclareCaptionLabelSeparator*{emdash}{~--- }
\captionsetup[figure]{labelsep=emdash,font=onehalfspacing,position=bottom}
%--------------------------------------

\usepackage{tempora}

%Листинги
%--------------------------------------
\usepackage{listings}
\lstset{
  basicstyle=\ttfamily\footnotesize, 
  %basicstyle=\footnotesize\AnkaCoder,        % the size of the fonts that are used for the code
  breakatwhitespace=false,         % sets if automatic breaks shoulbd only happen at whitespace
  breaklines=true,                 % sets automatic line breaking
  captionpos=t,                    % sets the caption-position to bottom
  inputencoding=utf8,
  frame=single,                    % adds a frame around the code
  keepspaces=true,                 % keeps spaces in text, useful for keeping indentation of code (possibly needs columns=flexible)
  keywordstyle=\bf,       % keyword style
  numbers=left,                    % where to put the line-numbers; possible values are (none, left, right)
  numbersep=5pt,                   % how far the line-numbers are from the code
  xleftmargin=25pt,
  xrightmargin=25pt,
  showspaces=false,                % show spaces everywhere adding particular underscores; it overrides 'showstringspaces'
  showstringspaces=false,          % underline spaces within strings only
  showtabs=false,                  % show tabs within strings adding particular underscores
  stepnumber=1,                    % the step between two line-numbers. If it's 1, each line will be numbered
  tabsize=2,                       % sets default tabsize to 8 spaces
  title=\lstname                   % show the filename of files included with \lstinputlisting; also try caption instead of title
}
%--------------------------------------

%%% Математические пакеты %%%
%--------------------------------------
\usepackage{amsthm,amsfonts,amsmath,amssymb,amscd}  % Математические дополнения от AMS
\usepackage{mathtools}                              % Добавляет окружение multlined
\usepackage[perpage]{footmisc}
%--------------------------------------

%--------------------------------------
%			НАЧАЛО ДОКУМЕНТА
%--------------------------------------

\begin{document}

%--------------------------------------
%			ТИТУЛЬНЫЙ ЛИСТ
%--------------------------------------
\begin{titlepage}
\thispagestyle{empty}
\newpage


%Шапка титульного листа
%--------------------------------------
\vspace*{-60pt}
\hspace{-65pt}
\begin{minipage}{0.3\textwidth}
\hspace*{-20pt}\centering
\includegraphics[width=\textwidth]{emblem}
\end{minipage}
\begin{minipage}{0.67\textwidth}\small \textbf{
\vspace*{-0.7ex}
\hspace*{-6pt}\centerline{Министерство науки и высшего образования Российской Федерации}
\vspace*{-0.7ex}
\centerline{Федеральное государственное бюджетное образовательное учреждение }
\vspace*{-0.7ex}
\centerline{высшего образования}
\vspace*{-0.7ex}
\centerline{<<Московский государственный технический университет}
\vspace*{-0.7ex}
\centerline{имени Н.Э. Баумана}
\vspace*{-0.7ex}
\centerline{(национальный исследовательский университет)>>}
\vspace*{-0.7ex}
\centerline{(МГТУ им. Н.Э. Баумана)}}
\end{minipage}
%--------------------------------------

%Полосы
%--------------------------------------
\vspace{-25pt}
\hspace{-35pt}\rule{\textwidth}{2.3pt}

\vspace*{-20.3pt}
\hspace{-35pt}\rule{\textwidth}{0.4pt}
%--------------------------------------

\vspace{1.5ex}
\hspace{-35pt} \noindent \small ФАКУЛЬТЕТ\hspace{80pt} <<Информатика и системы управления>>

\vspace*{-16pt}
\hspace{47pt}\rule{0.83\textwidth}{0.4pt}

\vspace{0.5ex}
\hspace{-35pt} \noindent \small КАФЕДРА\hspace{50pt} <<Теоретическая информатика и компьютерные технологии>>

\vspace*{-16pt}
\hspace{30pt}\rule{0.866\textwidth}{0.4pt}
  
\vspace{11em}

\begin{center}
\Large {\bf Летучка № 4} \\ 
\large {\bf по курсу <<Численные методы линейной алгебры>>} \\
\large <<Использование метода Холецкого для решения СЛАУ>> 
\end{center}\normalsize

\vspace{8em}


\begin{flushright}
  {Студент группы ИУ9-72Б Виленский С. Д. \hspace*{15pt}\\ 
  \vspace{2ex}
  Преподаватель Посевин Д. П.\hspace*{15pt}}
\end{flushright}

\bigskip

\vfill
 

\begin{center}
\textsl{Москва 2024}
\end{center}
\end{titlepage}
%--------------------------------------
%		КОНЕЦ ТИТУЛЬНОГО ЛИСТА
%--------------------------------------

\renewcommand{\ttdefault}{pcr}

\setlength{\tabcolsep}{3pt}
\newpage
\setcounter{page}{2}

\section{Задание}\label{Sect::task}

Вариант 1:
Реализовать алгоритм разложения положительно определенной симметричной матрицы на произведение верхнетреугольной на себя транпонированную.
Реализовать алгоритм решения СЛАУ при помощи полученного разложения.

\section{Результаты}\label{Sect::res}

Исходный код программы представлен в листингах~\ref{lst:code1}-~\ref{lst:code2}.

\begin{figure}[!htb]
\begin{lstlisting}[caption={Реализация алгоритма Холецкого},label={lst:code1}]
using LinearAlgebra
using Random

function generate_symmetric_positive_matrix(n::Int)::Matrix{Float64}
    A = rand(1.0:0.01:100.0, n, n)
    return A' * A
end

function triagonal_matrix_by_Cholesky(A::Matrix{Float64})::Matrix{Float64}
    n = size(A, 1)
    L = zeros(Float64, n, n)
    
    for i in 1:n
        L[i, i] = sqrt(A[i, i] - sum(L[i, p]^2 for p in 1:(i-1); init=0.0))

        for j in (i+1):n
            L[j, i] = (A[j, i] - sum(L[j, p] * L[i, p] for p in 1:(i-1); init=0.0)) / L[i, i]
        end
    end

    return L
end

function find_solve_with_Cholesky(A::Matrix{Float64}, f::Vector{Float64})::Vector{Float64}
    n = size(A, 1)
    L = triagonal_matrix_by_Cholesky(A)
    x = zeros(Float64, n)
    y = zeros(Float64, n)

    for i in 1:n
        y[i] = (f[i] - sum(L[i, k] * y[k] for k in 1:(i-1); init=0.0)) / L[i, i]
    end
\end{lstlisting}
\end{figure}
\begin{figure}[!htb]
\begin{lstlisting}[caption={Реализация алгоритмов оценки источников погрешностей},label={lst:code2}]
    for i in n:-1:1
        x[i] = (y[i] - sum(L[k, i] * x[k] for k in (i+1):n; init=0.0)) / L[i, i]
    end

    return x
end

n = 4
A = generate_symmetric_positive_matrix(n)
x = rand(Float64, n)
x_ = find_solve_with_Cholesky(A, A * x)

print("Error: ", norm(x - x_))

\end{lstlisting}
\end{figure}

Результат работы программы представлен в листинге ~\ref{lst:code3}.

\begin{figure}[!htb]
\begin{lstlisting}[caption={Результат работы программы},label={lst:code3}]
Error: 3.2870149053529396e-15
\end{lstlisting}
\end{figure}

\section{Выводы}\label{Sect::conclusion}

Алгоритм Холецкого является эффективным способом решения СЛАУ с симметричной положительно определенной матрицей коэффициентов.

\end{document}
