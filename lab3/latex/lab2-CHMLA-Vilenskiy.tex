\documentclass[a4paper, 14pt]{extarticle}

% Поля
%--------------------------------------
\usepackage{geometry}
\geometry{a4paper,tmargin=2cm,bmargin=2cm,lmargin=3cm,rmargin=1cm}
%--------------------------------------


%Russian-specific packages
%--------------------------------------
\usepackage[T2A]{fontenc}
\usepackage[utf8]{inputenc} 
\usepackage[english, main=russian]{babel}
%--------------------------------------

\usepackage{textcomp}

% Красная строка
%--------------------------------------
\usepackage{indentfirst}               
%--------------------------------------             


%Graphics
%--------------------------------------
\usepackage{graphicx}
\graphicspath{ {./images/} }
\usepackage{wrapfig}
%--------------------------------------

% Полуторный интервал
%--------------------------------------
\linespread{1.3}                    
%--------------------------------------

%Выравнивание и переносы
%--------------------------------------
% Избавляемся от переполнений
\sloppy
% Запрещаем разрыв страницы после первой строки абзаца
\clubpenalty=10000
% Запрещаем разрыв страницы после последней строки абзаца
\widowpenalty=10000
%--------------------------------------

%Списки
\usepackage{enumitem}

%Подписи
\usepackage{caption} 

%Гиперссылки
\usepackage{hyperref}

\hypersetup {
	unicode=true
}

%Рисунки
%--------------------------------------
\DeclareCaptionLabelSeparator*{emdash}{~--- }
\captionsetup[figure]{labelsep=emdash,font=onehalfspacing,position=bottom}
%--------------------------------------

\usepackage{tempora}

%Листинги
%--------------------------------------
\usepackage{listings}
\lstset{
  basicstyle=\ttfamily\footnotesize, 
  %basicstyle=\footnotesize\AnkaCoder,        % the size of the fonts that are used for the code
  breakatwhitespace=false,         % sets if automatic breaks shoulbd only happen at whitespace
  breaklines=true,                 % sets automatic line breaking
  captionpos=t,                    % sets the caption-position to bottom
  inputencoding=utf8,
  frame=single,                    % adds a frame around the code
  keepspaces=true,                 % keeps spaces in text, useful for keeping indentation of code (possibly needs columns=flexible)
  keywordstyle=\bf,       % keyword style
  numbers=left,                    % where to put the line-numbers; possible values are (none, left, right)
  numbersep=5pt,                   % how far the line-numbers are from the code
  xleftmargin=25pt,
  xrightmargin=25pt,
  showspaces=false,                % show spaces everywhere adding particular underscores; it overrides 'showstringspaces'
  showstringspaces=false,          % underline spaces within strings only
  showtabs=false,                  % show tabs within strings adding particular underscores
  stepnumber=1,                    % the step between two line-numbers. If it's 1, each line will be numbered
  tabsize=2,                       % sets default tabsize to 8 spaces
  title=\lstname                   % show the filename of files included with \lstinputlisting; also try caption instead of title
}
%--------------------------------------

%%% Математические пакеты %%%
%--------------------------------------
\usepackage{amsthm,amsfonts,amsmath,amssymb,amscd}  % Математические дополнения от AMS
\usepackage{mathtools}                              % Добавляет окружение multlined
\usepackage[perpage]{footmisc}
%--------------------------------------

%--------------------------------------
%			НАЧАЛО ДОКУМЕНТА
%--------------------------------------

\begin{document}

%--------------------------------------
%			ТИТУЛЬНЫЙ ЛИСТ
%--------------------------------------
\begin{titlepage}
\thispagestyle{empty}
\newpage


%Шапка титульного листа
%--------------------------------------
\vspace*{-60pt}
\hspace{-65pt}
\begin{minipage}{0.3\textwidth}
\hspace*{-20pt}\centering
\includegraphics[width=\textwidth]{emblem}
\end{minipage}
\begin{minipage}{0.67\textwidth}\small \textbf{
\vspace*{-0.7ex}
\hspace*{-6pt}\centerline{Министерство науки и высшего образования Российской Федерации}
\vspace*{-0.7ex}
\centerline{Федеральное государственное бюджетное образовательное учреждение }
\vspace*{-0.7ex}
\centerline{высшего образования}
\vspace*{-0.7ex}
\centerline{<<Московский государственный технический университет}
\vspace*{-0.7ex}
\centerline{имени Н.Э. Баумана}
\vspace*{-0.7ex}
\centerline{(национальный исследовательский университет)>>}
\vspace*{-0.7ex}
\centerline{(МГТУ им. Н.Э. Баумана)}}
\end{minipage}
%--------------------------------------

%Полосы
%--------------------------------------
\vspace{-25pt}
\hspace{-35pt}\rule{\textwidth}{2.3pt}

\vspace*{-20.3pt}
\hspace{-35pt}\rule{\textwidth}{0.4pt}
%--------------------------------------

\vspace{1.5ex}
\hspace{-35pt} \noindent \small ФАКУЛЬТЕТ\hspace{80pt} <<Информатика и системы управления>>

\vspace*{-16pt}
\hspace{47pt}\rule{0.83\textwidth}{0.4pt}

\vspace{0.5ex}
\hspace{-35pt} \noindent \small КАФЕДРА\hspace{50pt} <<Теоретическая информатика и компьютерные технологии>>

\vspace*{-16pt}
\hspace{30pt}\rule{0.866\textwidth}{0.4pt}
  
\vspace{11em}

\begin{center}
\Large {\bf Лабораторная работа № 3} \\ 
\large {\bf по курсу <<Численные методы линейной алгебры>>} \\
\large <<Оценки источников погрешностей решения СЛАУ методом Гаусса>> 
\end{center}\normalsize

\vspace{8em}


\begin{flushright}
  {Студент группы ИУ9-72Б Виленский С. Д. \hspace*{15pt}\\ 
  \vspace{2ex}
  Преподаватель Посевин Д. П.\hspace*{15pt}}
\end{flushright}

\bigskip

\vfill
 

\begin{center}
\textsl{Москва 2024}
\end{center}
\end{titlepage}
%--------------------------------------
%		КОНЕЦ ТИТУЛЬНОГО ЛИСТА
%--------------------------------------

\renewcommand{\ttdefault}{pcr}

\setlength{\tabcolsep}{3pt}
\newpage
\setcounter{page}{2}

\section{Задание}\label{Sect::task}

Реализовать формулы оценок ошибок округления и исходных данных.

\section{Результаты}\label{Sect::res}

Исходный код программы представлен в листингах~\ref{lst:code1}-~\ref{lst:code5}.

\begin{figure}[!htb]
\begin{lstlisting}[caption={Реализация алгоритмов оценки источников погрешностей},label={lst:code1}]
using LinearAlgebra
using Random

## Generating input data

function generate_matrix(n::Int64)::Matrix{Float64}
    return rand(-100:0.01:100, n, n)
end
function generate_matrix_delta(n::Int64)::Matrix{Float64}
    return rand(-10:0.01:10, n, n)
end

function generate_vector(n::Int64)::Vector{Float64}
    return rand(-100:0.01:100, n)
end
function generate_vector_delta(n::Int64)::Vector{Float64}
    return rand(-10:0.01:10, n)
end

## Norms

function euclidean_norm(A)::Float64
    return sqrt(sum(abs2, A))
end

function uniform_norm(x::Vector{Float64})::Float64
    return maximum(abs.(x))
end

function uniform_norm(x::Matrix{Float64})::Float64
    return maximum(sum(abs, x, dims=2))
end

## Condition number of a matrix

function get_condition_number(A::Matrix{Float64}, _norm::Function)::Float64
    return _norm(inv(A)) * _norm(A)
end

## Growth rate of matrix elements

function gauss_growth_factor(A::Matrix{Float64})::Float64
    n = size(A, 1)
\end{lstlisting}
\end{figure}
\begin{figure}[!htb]
\begin{lstlisting}[caption={Реализация алгоритмов оценки источников погрешностей},label={lst:code2}]
    max_initial = maximum(abs.(A))
    max_during = max_initial

    A_work = copy(A)

    for k in 1:n-1
        for i in k+1:n
            if A_work[k, k] != 0
                factor = A_work[i, k] / A_work[k, k]
                for j in k:n
                    A_work[i, j] -= factor * A_work[k, j]
                end
            end
        end
        max_during = max(max_during, maximum(abs.(A_work)))
    end
    
    return max_during / max_initial
end

## Error estimates

function error_rounding(
    A::Matrix{Float64},
    p::Int64,
    t::Int64,
    _norm::Function
)::Float64
    nu_A = get_condition_number(A, _norm)
    n = size(A, 1)
    g_A = gauss_growth_factor(A)
    return nu_A * n * g_A / p ^ t
end

function error_input_data(
    A::Matrix{Float64}, delta_A::Matrix{Float64},
    f::Vector{Float64}, delta_f::Vector{Float64},
    _norm::Function
)::Float64
    nu_A = get_condition_number(A, _norm)
    error_A = _norm(delta_A) / _norm(A)
    error_f = _norm(delta_f) / _norm(f)
    return nu_A * (error_A + error_f)
end

## Solution error

function error_result(
    A::Matrix{Float64}, delta_A::Matrix{Float64},
    f::Vector{Float64}, delta_f::Vector{Float64},
    _norm::Function
)::Float64
    x = A \ f
    delta_x = (A + delta_A) \ (f + delta_f) - x
    return _norm(delta_x) / _norm(x)
end
\end{lstlisting}
\end{figure}
\begin{figure}[!htb]
\begin{lstlisting}[caption={Реализация алгоритмов оценки источников погрешностей},label={lst:code3}]
## Testing

A = [
    100. 99.;
    99. 98.
]
delta_A = [
    0. 0.;
    0. 0.
]
f = [
    199.,
    197.
]
delta_f = [
    -.01,
    .01
]

print("Relative error of input data: ", error_input_data(A, delta_A, f, delta_f, uniform_norm))
print("\nRelative error of result: ", error_result(A, delta_A, f, delta_f, uniform_norm))

A = generate_matrix(5)
delta_A = generate_matrix_delta(5)
f = generate_vector(5)
delta_f = generate_vector_delta(5)

_norm = euclidean_norm

print("Relative error of rounding: ", error_rounding(A, 2, 11, _norm))
print("\nRelative error of input data: ", error_input_data(A, delta_A, f, delta_f, _norm))
print("\nRelative error of result: ", error_result(A, delta_A, f, delta_f, _norm))
\end{lstlisting}
\end{figure}

Результат работы программы представлен в листинге ~\ref{lst:code4}.

\begin{figure}[!htb]
\begin{lstlisting}[caption={Результат работы программы},label={lst:code4}]
Relative error of input data: 1.9899999999989821
Relative error of result: 1.9899999999940312

Relative error of rounding: 0.11516827877965007
Relative error of input data: 1.32573555646218
Relative error of result: 0.1490879688109449
\end{lstlisting}
\end{figure}

\section{Выводы}\label{Sect::conclusion}

Проанализировав графики зависимостей ошибок разных методов от диагонального доминирования матриц можно сделать вывод о том, что самым оптимальным является модификация метода Гаусса перестановкой по строкам и столбцам, модификации метода Гаусса перестановкой по строкам или по столбцам относительно имеет близкую погрешность и классический метод Гаусса среди прочих имеет наибольшую относительную ошибку.

\end{document}
